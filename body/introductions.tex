% !Mode:: "TeX:UTF-8"
\chapter{关于~\LaTeX{}~}
\label{chap:introduction}
\section{简介}
~\LaTeX{}~ \footnote{以下介绍均来自\href{http://baike.baidu.com/link?url=lUEyKcwLy5H-qEgRZlHI2H1aLND2Yd2x8vuTfJhXNq3Y5qhDc3oxRzANR0WUyI5A8qohsqrtoE0a02I20xcena}{百度百科}}(音译“拉泰赫”)是一种基于~\TeX{}~的排版系统,由美国计算机学家莱斯利·兰伯特(Leslie Lamport)在20世纪80年代初期开发,利用这种格式,即使使用者没有排版和程序设计的知识也可以充分发挥由~\TeX{}~所提供的强大功能,能在几天,甚至几小时内生成很多具有书籍质量的印刷品。对于生成复杂表格和数学公式,这一点表现得尤为突出。因此它非常适用于生成高印刷质量的科技和数学类文档。这个系统同样适用于生成从简单的信件到完整书籍的所有其他种类的文档。
\section{释义}
【正式名称】:~\LaTeX{}~


【纯文本名称】:LaTeX


【概述】:~\LaTeX{}~使用~\TeX{}~作为它的格式化引擎,当前的版本是LaTeX2e。Leslie Lamport开发的~\LaTeX{}~是当今世界上最流行和使用最为广泛的~\TeX{}~宏集。它构筑在PlainTeX的基础之上,并加进了很多的功能以使得使用者可以更为方便的利用~\TeX{}~的强大功能。使用LaTeX基本上不需要使用者自己设计命令和宏等,因为~\LaTeX{}~已经替你做好了。因此,即使使用者并不是很了解~\TeX{}~,也可以在短短的时间内生成高质量的文档。对于生成复杂的数学公式,~\LaTeX{}~表现的更为出色。~\LaTeX{}~自从八十年代初问世以来,也在不断的发展.最初的正式版本为2.09,在经过几年的发展之后,许多新的功能,机制被引入到~\LaTeX{}~中。在享受这些新功能带来的便利的同时,它所伴随的副作用也开始显现,这就是不兼容性。标准的~\LaTeX{}~ 2.09引入了“新字体选择框架”(NFSS)的~\LaTeX{}~、SLiTeX,AMS-LaTeX等等,相互之间并不兼容.这给使用者和维护者都带来很大的麻烦。为结束这种糟糕的状况,Frank、Mittel、bach等人成立了ATeX3项目小组,目标是建立一个最优的,有效的,统一的,标准的命令集合。即得到~\LaTeX{}~的一个新版本3.这是一个长期目标,向这个目标迈出第一步就是在1994年发布的LaTeX2ε。~LaTeX2e采用了NFSS作为标准,加入了很多新的功能,同时还兼容旧~\LaTeX{}~ 2.09。LaTeX2e每6个月更新一次,修正发现的错误并加入前,LaTeX2e将是标准的。
\section{历史}
\subsection{~\TeX{}~格式}
最基本的~\TeX{}~程序只是由一些很原始的命令组成,它们可以完成简单的排版操作和程序设计功能。然而,~\TeX{}~也允许用这些原始命令定义一些更复杂的高级命令。这样就可以利用低级的块结构,形成一个用户界面相当友好的环境。


在处理器运行期间,该程序首先读取所谓的格式文件,其中包含各种以原始语言写成的高级命令,也包含分割单词的连字号安排模式。接着处理程序就处理源文件,其中包含要处理的真正文本,以及在格式文件中已定义了的格式命令。


创建新格式是一件需要由具有丰富知识的程序员来做的事情。把定义写到一个源文件中,这个文件接着被一个名叫iniTeX的特殊版本的~\TeX{}~程序处理。它采用一种紧凑的方式存贮这些新格式,这样就可以被通常~\TeX{}~程序很快地读取。
\subsection{PlainTeX}
Knuth设计了一个名叫PlainTeX的基本格式,以与低层次的原始~\TeX{}~呼应。这种格式是用~\TeX{}~处理文本时相当基本的部分,以致于我们有时都分不清到底哪条指令是真正的处理程序~\TeX{}~的原始命令,哪条是PlainTeX格式的。大多数声称只使用~\TeX{}~的人,实际上指的是只用PlainTeX。


PlainTeX也是其它格式的基础,这进一步加深了很多人认为~\TeX{}~和PlainTeX是同一事物的印象。


PlainTeX的重点还只是在于如何排版的层次上,而不是从一位作者的观点出发。对它的深层功能的进一步发掘,需要相当丰富的编程技巧。因此它的应用就局限于高级排版和程序设计人员。
\subsection{~\LaTeX{}~}
Leslie Lamport开发的~\LaTeX{}~是当今世界上最流行和使用最为广泛的~\TeX{}~格式。它构筑在 PlainTeX的基础之上,并加进了很多的功能以使得使用者可以更为方便的利用~\TeX{}~的强大功能。使用~\LaTeX{}~基本上不需要使用者自己设计命令和宏等,因为~\LaTeX{}~已经替你做好了。因此,即使使用者并不是很了解~\TeX{}~,也可以在短短的时间内生成高质量的文档。对于生成复杂的数学公式,~\LaTeX{}~表现的更为出色。


~\LaTeX{}~自从二十世纪八十年代初问世以来,也在不断的发展。最初的正式版本为 2.09,在经过几年的发展之后,许多新的功能,机制被引入到~\LaTeX{}~中。在享受这些新功能带来的便利的同时,它所伴随的副作用也开始显现,这就是不兼容性。标准的~\LaTeX{}~2.09,引入了“新字体选择框架”(NFSS)的~\LaTeX{}~,SLiTeX,AMSLaTeX 等等,相互之间并不兼容。这给使用者和维护者都带来很大的麻烦。
\subsection{LaTeX2e}
为结束这种糟糕的状况,Frank Mittelbach等人成立了~\LaTeX{}~3项目小组,目标是建立一个最优的,有效的,统一的,标准的命令集合。即得到~\LaTeX{}~的一个新版本3。这是一个长期目标,向这个目标迈出第一步就是在1994年发布的LaTeX2e。LaTeX2e采用了NFSS作为标准,加入了很多新的功能,同时还兼容旧的~\LaTeX{}~2.09。LaTeX2e每6个月更新一次,修正发现的错误并加入一些新的功能。在~\LaTeX{}~3最终完成之前,~LaTeX2e将是标准的~\LaTeX{}~版本。
\subsection{~\LaTeX{}~各版本关系}
MiKTeX、fpTeX、teTeX、CTeX是什么关系?~\TeX{}~ 在不同的硬件和操作系统上有不同的实现版本。这就像C语言,在不同的操作系统中有不同的编译系统,例如Linux 下的gcc,Windows 下的Visual C++ 等。有时,一种操作系统里也会有好几种的~\TeX{}~系统。目前常见的Unix/Linux下的~\TeX{}~系统是TeXlive,Windows下则有MiKTeX和fpTeX。CTeX指的是CTeX中文套装的简称,是把MiKTeX和一些常用的相关工具,如GSview,WinEdt 等包装在一起制作的一个简易安装程序,并对其中的中文支持部分进行了配置,使得安装后马上就可以使用中文。