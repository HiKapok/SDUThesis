% !Mode:: "TeX:UTF-8"

\makeatletter

\cheading{山东大学学士学位论文}      		% 设置正文的页眉,需要填上对应的毕业年份
\ctitle{基于纹理特征的遥感图像分类方法研究} 				% 封面用论文标题,自己可手动断行
\caffil{控制科学与工程学院} 				% 学院名称
\csubject{自动化}   		% 专业名称
\cgrade{2012~级}            		% 年级
\cauthor{XXX}         			% 学生姓名
\cnumber{201200xxxxxx}        		% 学生学号
\csupervisor{XXX}      	% 导师姓名
\crank{副教授}              		% 导师职称
\cdate{\the\year~年~\the\month~月~\the\day~日}

\newlength{\@title@width}
% 定义封面
\setcounter{page}{1}
\def\makecover{
\phantomsection
\pdfbookmark[-1]{\@ctitle}{ctitle}
\begin{titlepage}

\begin{center}
	\vspace*{10pt}
  	\begin{figure}[h]
  		\centering
  		\includegraphics[width=0.5\textwidth]{figures/sdupic}
  	\end{figure}
  	\vspace*{-30pt}
  	
  	\hei\cuhao{\bf{毕业论文(设计)\\}}
  	
  	\vspace{42pt}

  	\song\yihao{\bf{
  		\begin{tabular}{lc}
    	论文题目:\;	& 基于纹理特征的遥感图像  \\
					& 分类方法研究 \\
  		\end{tabular}
  	}}  
  
	\vspace{100pt}

  	\setlength{\@title@width}{7cm}
  	{
  		\fontsize{16pt}{24pt}\selectfont\kai{\bf{
  		\begin{tabular}{lc}
    		学\qquad 院 &  \underline{\makebox[\@title@width][c]{\@caffil}} \\
   			专\qquad 业 &  \underline{\makebox[\@title@width][c]{\@csubject}} \\
    		年\qquad 级 &  \underline{\makebox[\@title@width][c]{\@cgrade}}\\
    		姓\qquad 名 &  \underline{\makebox[\@title@width][c]{\@cauthor}} \\
    		学\qquad 号 &  \underline{\makebox[\@title@width][c]{\@cnumber}}\\
    		指导教师     &  \underline{\makebox[\@title@width][c]{\@csupervisor}} \\
  			\end{tabular}
  		}}
 	}
 	\vspace{40pt}

	\fontsize{15pt}{22pt}\selectfont\kai{\textbf{\@cdate}}
\end{center}

\end{titlepage}

\clearpage
\pagestyle{empty}
\mbox{}
\clearpage
}

\cabstract{
这里是中文摘要。格式要求:(1)居中打印“摘要”二字(三号黑体),二字之间空一格。(2)“摘要”二字下空一行打印摘要内容(小四号宋体),摘要内容每段开头缩进两个字。切忌将应在引言中出现的内容(如研究背景等)写入摘要,一般也不要对论文内容作诠释和评论(尤其是自我评价)。摘要中尽量少用特殊字符以及由特殊字符组成的数学表达式。
}

\ckeywords{摘要内容下空一行,顶格位置打印“关键词”三字(小四号黑体),其后为关键词(小四号宋体)。每一关键词之间用逗号隔开,最后一个关键词后不打标点符号。关键字总数在3-7个为宜。}

\eabstract{
This is the English Abstract.The standard format:(1)write "ABSTRACT" with the size of 16pt(three) and center it horizontally,Skip two lines and type the abstract content with the font Times NewRoman(12pt);(2)Leave four blank at the begin of each 
paragraph.}

\ekeywords{Skip one line down and write "KEYWORDS",then put the keywords,separated by commas(no punctuation at the end).the number of keywords varies from 3 to 7.}

\makeatother

\makecover

%%%%%%%%%%%%%%%%%%%   Abstract and Keywords  %%%%%%%%%%%%%%%%%%%%%%%
\makeatletter
\setcounter{page}{1}
\pagenumbering{Roman}
\markboth{摘~要}{摘~要}
\addcontentsline{toc}{chapter}{摘\quad 要}
\chapter*{\centering\sanhao\hei 摘\quad 要}
\song\xiaosi\selectfont
\vbox{}
\vspace*{-5pt}
\@cabstract
\vspace{\baselineskip}

\hangafter=1\hangindent=52.3pt\noindent
{\hei\xiaosi 关键词:} \@ckeywords

\clearpage
\pagestyle{empty}
\mbox{}

%%%%%%%%%%%%%%%%%%%   English Abstract  %%%%%%%%%%%%%%%%%%%%%%%%%%%%%%
\clearpage
\setcounter{page}{2}
\markboth{ABSTRACT}{ABSTRACT}
\addcontentsline{toc}{chapter}{ABSTRACT}
\chapter*{\centering\sanhao{\bf{ABSTRACT}}}
\vspace{\baselineskip}
\@eabstract
\vspace{\baselineskip}

\hangafter=1\hangindent=60pt\noindent
{\textbf{Keywords:}} \@ekeywords

\clearpage
\pagestyle{empty}
\mbox{}
\clearpage

\fancypagestyle{plain}{
\fancyhf{}
\renewcommand{\headrulewidth}{0 pt}
\fancyfoot[C]{\song\xiaowu~\thepage~}
}
%%%%%%%%%%   目录   %%%%%%%%%%
\defaultfont
\setcounter{page}{1}
\pagenumbering{arabic}
\titleformat{\chapter}{\centering\xiaoer\hei}{\chaptername}{\sepchapter}{} % 设置目录两字的格式

\pdfbookmark[0]{目~~录}{mulu}
\pagestyle{plain}

\tableofcontents                                     % 中文目录

\cleardoublepage
\makeatother

\mainmatter\defaultfont\sloppy\raggedbottom

\makeatletter

\fancypagestyle{plain}{                              % 设置开章页眉页脚风格
    \fancyhf{}
    \fancyhead[C]{\song\wuhao \@cheading}            % 首页页眉格式
    \fancyfoot[C]{\song\xiaowu ~\thepage~}           % 首页页脚格式
    \renewcommand{\headrulewidth}{0.5pt}
    \renewcommand{\footrulewidth}{0pt}
}
% 设置其余正文部分页眉页脚风格
\pagestyle{plain}
\fancyhf{}
\fancyhead[C]{\song\wuhao \@cheading}
\fancyfoot[C]{\song\xiaowu ~\thepage~}
\renewcommand{\headrulewidth}{0.5pt}
\renewcommand{\footrulewidth}{0pt}
\makeatother

\setcounter{page}{1}                                 % 单独从 1 开始编页码
\titleformat{\chapter}{\centering\sanhao\hei}{\chaptername}{\sepchapter}{} % 恢复chapter标题格式要求

